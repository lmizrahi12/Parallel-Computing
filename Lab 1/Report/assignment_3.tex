%%%%%%%%%%%%%%%%%%%%%%%%%%%%%%%%%%%%%%%%%
% Short Sectioned Assignment
% LaTeX Template
% Version 1.0 (5/5/12)
%
% This template has been downloaded from:
% http://www.LaTeXTemplates.com
%
% Original author:
% Frits Wenneker (http://www.howtotex.com)
%
% License:
% CC BY-NC-SA 3.0 (http://creativecommons.org/licenses/by-nc-sa/3.0/)
%
%%%%%%%%%%%%%%%%%%%%%%%%%%%%%%%%%%%%%%%%%

%----------------------------------------------------------------------------------------
%	PACKAGES AND OTHER DOCUMENT CONFIGURATIONS
%----------------------------------------------------------------------------------------

\documentclass[paper=a4, fontsize=11pt]{scrartcl} % A4 paper and 11pt font size

\usepackage[T1]{fontenc} % Use 8-bit encoding that has 256 glyphs
\usepackage{fourier} % Use the Adobe Utopia font for the document - comment this line to return to the LaTeX default
\usepackage[english]{babel} % English language/hyphenation
\usepackage{amsmath,amsfonts,amsthm} % Math packages

\usepackage{lipsum} % Used for inserting dummy 'Lorem ipsum' text into the template

\usepackage{graphicx}

\usepackage{sectsty} % Allows customizing section commands
\allsectionsfont{\centering \normalfont\scshape} % Make all sections centered, the default font and small caps

\usepackage{fancyhdr} % Custom headers and footers
\pagestyle{fancyplain} % Makes all pages in the document conform to the custom headers and footers
\fancyhead{} % No page header - if you want one, create it in the same way as the footers below
\fancyfoot[L]{} % Empty left footer
\fancyfoot[C]{} % Empty center footer
\fancyfoot[R]{\thepage} % Page numbering for right footer
\renewcommand{\headrulewidth}{0pt} % Remove header underlines
\renewcommand{\footrulewidth}{0pt} % Remove footer underlines
\setlength{\headheight}{13.6pt} % Customize the height of the header

\numberwithin{equation}{section} % Number equations within sections (i.e. 1.1, 1.2, 2.1, 2.2 instead of 1, 2, 3, 4)
\numberwithin{figure}{section} % Number figures within sections (i.e. 1.1, 1.2, 2.1, 2.2 instead of 1, 2, 3, 4)
\numberwithin{table}{section} % Number tables within sections (i.e. 1.1, 1.2, 2.1, 2.2 instead of 1, 2, 3, 4)

\setlength\parindent{0pt} % Removes all indentation from paragraphs - comment this line for an assignment with lots of text

%----------------------------------------------------------------------------------------
%	TITLE SECTION
%----------------------------------------------------------------------------------------

\newcommand{\horrule}[1]{\rule{\linewidth}{#1}} % Create horizontal rule command with 1 argument of height

\title{	
\normalfont \normalsize 
\horrule{0.5pt} \\[0.4cm] % Thin top horizontal rule
\huge Parallel Computing - Lab 1 \\ % The assignment title
\horrule{1pt} \\[0.5cm] % Thick bottom horizontal rule
}

\author{Liron Mizrahi - 708810} % Your name

\begin{document}

\maketitle % Print the title

%----------------------------------------------------------------------------------------
%	ARRAY ACCESSING
%----------------------------------------------------------------------------------------

\section{Accessing Multi-Dimensional Arrays}

When accessing multi-dimensional array, it can be accessed column-wise or row-wise. Each row will be stored in contiguous memory blocks and may be seperate from other row's memory. So when the array is accessed column-wise, it will have to load every row into cache seperately to access one element. Whereas if it is accessed row-wise, each row can be loaded into cache and every value can be read from cache before loading the next row.

\begin{figure}[!h]
	\centering
	\begin{minipage}[b]{0.4\textwidth}
		\includegraphics[width=\linewidth]{"../../Coms/PC/Lab 1/Graphs/RowVsColumn_row"}
		\caption{Row-wise}
	\end{minipage}
	\hfill
	\begin{minipage}[b]{0.4\textwidth}
		\includegraphics[width=\linewidth]{"../../Coms/PC/Lab 1/Graphs/RowVsColumn_column"}
	\caption{Column-wise}
	\end{minipage}
\end{figure}

It can be seen from the two graphs above that accessing arrays row-wise is significantly faster than accessing it column-wise.

%----------------------------------------------------------------------------------------
%	THREAD OUTPUT
%----------------------------------------------------------------------------------------

\section{Display Output From Threads}
If the amount of threads being used is more than the amount of processing units available then some threads will have to be scheduled. The scheduler of the OS being used will determine which threads will run and for how long. This makes the effects of threads unpredictable. The scheduler will keep switching threads which results in the same program giving different outputs.

%----------------------------------------------------------------------------------------
%	DOT PRODUCT
%----------------------------------------------------------------------------------------

\section{Vector Dot Product}

%----------------------------------------------------------------------------------------

\subsection{Effect of Varying the Number of Threads}
By increasing the number of threads, some computation can be done at the same time. However if the amount of threads being used is more than the amount of processing units available then there will be scheduling and not all computation will be done in parallel. In the case of calculating the dot product, the actual computations are simple and therefore the startup cost of the threads will outweigh the calculations done per thread. 

	\begin{figure}[!h]
		\centering
		\includegraphics[width=0.6\linewidth]{"../../Coms/PC/Lab 1/Graphs/dot-prod_num-threads"}
		\caption{Effect of increasing the number of threads for array sizes of 5000}
	\end{figure}

As seen in the graph above, the effect of increasing the threads is not always going to result in better performance. It can also be seen that for very small amount of threads, close to the amount of processing units actually available in the system, the time decreases as all the threads are running in parallel with minimal scheduling.

\newpage

%----------------------------------------------------------------------------------------

\subsection{Effect of Varying the Size of the Vectors}
As the size of the arrays increase, the amount of computations scales proportionally. The work load would be distributed between the threads as to minimize any idling.

	\begin{figure}[!h]
		\centering
		\includegraphics[width=0.6\linewidth]{"../../Coms/PC/Lab 1/Graphs/dot-prod_array-sizes"}
		\caption{Effect of increasing the size of the arrays}
	\end{figure}

As expected the time taken to complete the dot product will increase proportionally at a linear rate.

%----------------------------------------------------------------------------------------

\subsection{Parallelizing Random Number Generation}
 Theoretically parallelising the random number generator should increase the performance. However, the startup cost of the threads and the fact that these programs were run in a virtual machine will cause more scheduling to occur and actually give worse time practically.
 
	\begin{figure}[!h]
		\centering
		\includegraphics[width=0.6\linewidth]{"../../Coms/PC/Lab 1/Graphs/dot-prod_parallel-rng"}
		\caption{Effect of parallelising the RNG}
	\end{figure}

As seen in the graph, the times taken are worse than a sequential RNG in this instance.

%----------------------------------------------------------------------------------------

\end{document}